\documentclass[a4paper,titlepage]{jlreq}

\usepackage{graphicx}
\usepackage{listings}
\lstset{
  basicstyle=\small\ttfamily,
  identifierstyle=\small,
  ndkeywordstyle=\small,
  tabsize=4,
  frame={shadowbox},
  frameround={ffff},
  breaklines=true,
  columns=[l]{fullflexible},
  numbers=left,
  numbersep=5pt,
  numberstyle=\scriptsize,
  stepnumber=1,
  lineskip=-0.5ex
}
\renewcommand{\lstlistingname}{ソースコード} % キャプション名の変更
\usepackage{hyperref}
\usepackage{here}

\title{レポートサンプル}
\author{201811377 広瀬智之}
\date{\today}

\begin{document}

\maketitle

\section{実験の目的}
本実験の目的は,モンテカルロ法による定積分\footnote{ヒットミス法ともいう}を用いて半径1の円の面積を求めることである.

\section{実験方法}
C言語のプログラムを用いて検証する.
具体的には,以下の2つを求める.

\begin{enumerate}
    \item{不等式$x^{2}+y^{2}\leq1$を満たす乱数の組$(x,y)$を500組発生させ,グラフを作成する}
    \item{不等式$-1\leq x \leq1, -1\leq y \leq1$を満たす乱数の組$(x,y)$を発生させ,表\ref{tab}を作成する}
\end{enumerate}

\begin{table}[H]
\begin{center}
\caption{求める表}
\begin{tabular}{|l|c|c|c|}
\hline
乱数の組の数               & 100 & 1000 & 10000 \\ \hline
面積の平均値(10試行)         &     &      &       \\ \hline
$((\pi - 面積の平均値) / \pi) * 4$ &     &      &       \\ \hline
\end{tabular}
\label{tab}
\end{center}
\end{table}

\section{実験結果}
\subsection{実験1}
プログラム\ref{ex1}を用いて乱数を発生させ,gnuplotを用いてグラフ\ref{graph}を作成した.

%%\begin{figure}[H]
%%    \centering
%%    \caption{グラフ}
%%    \includegraphics[clip,width=0.9\linewidth]{../graph01.pdf}
%%    \label{graph}
%%\end{figure}

\subsection{実験2}
プログラム\ref{ex2}を用いて乱数を発生させ,表\ref{tab:ok}を作成した.

\begin{table}[H]
\begin{center}
\caption{求める表}
\begin{tabular}{|l|c|c|c|}
\hline
乱数の組の数               & 100 & 1000 & 10000 \\ \hline
面積の平均値(10試行)         & 0.779000 & 0.791600 & 0.785830  \\ \hline
$((\pi - 面積の平均値) / \pi) * 4$ & 3.008146 & 2.992103 & 2.999449  \\ \hline
\end{tabular}
\label{tab:ok}
\end{center}
\end{table}

\section{考察}
結果は概ね3に収束した.
サンプル数を増やしていけば円周率にさらに近づくと考えられる.


\section{プログラムリスト}

%%\lstinputlisting[label=ex1,language=c,caption=実験1]{../ex1.c}
%%\lstinputlisting[label=ex2,language=c,caption=実験2]{../ex2.c}

\end{document}
